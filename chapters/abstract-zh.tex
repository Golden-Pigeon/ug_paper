% 随着深度学习技术的发展,近年来,基于深度学习模型的图像超分辨率技术取得了极大的成功,并在多数仿真数据集中取得了远超传统方法的的成绩。但是,将这些超分辨率方法应用于真实低分辨率图像时,其超分辨率效果相比于仿真图像有显著的下降,同时会产生伪影。其主要原因是仿真数据集通常假设图像通过高斯模糊和双三次下采样,退化过程已知,而真实低分辨率图像往往拥有未知的退化过程,包括未知的噪声、模糊、图像压缩等,这造成其与仿真低分辨率图像拥有较高的域差距,从而影响了训练出来的模型的超分辨率效果。

% 为了应对此问题,可以通过使用真实成对的数据集训练模型,如City100、RealSR等。然而,真实成对数据集的收集工作往往费力且昂贵,此外,由于不同设备采集到的低分辨率图像间的域差距,使用此方法训练的超分辨率模型往往不能应用到其他数据集中。因此,研究人员提出了许多基于无监督学习的超分辨率方法,即结合图像退化模型,使用非成对数据集训练一个降采样网络,并通过此网络大量生成伪成对的数据集。使用生成的数据集即可对超分辨率模型进行监督式的训练,从而避免了成对数据集的采集成本,并提高了模型的泛化性。

% 近期,一份对图像退化过程不确定性进行建模的工作USR-DU取得了良好的效果,并在多个数据集中超过了当前最先进的模型。该模型在模糊性较低的真实低分辨率图像上拥有优异的表现,但在模糊性较高的真实低分辨率图像上表现较差。为了解决此问题,本文提出了改进模型\textbf{UBSR-DU}。本文首先比较了USR-DU的降采样网络生成的低分辨率图像与真实图像之间的差距,发现该差距出于对图像退化中的模糊过程的忽视。此后,为了将模糊信息融入到降采样网络中,分析了多种模糊核估计方法,综合考虑其设计思想与估计效果,最后采用了基于空间变化核估计的MANet。最后,设计了对训练批次中低分辨率图像块进行模糊核估计,并将模糊核应用于降采样网络的参考图像的生成过程中的训练流程,从而成功地在降采样网络中融入了模糊信息。此外,考虑到本文中超分网络的测试集中具有较多的文字,而训练集高分辨率图像中文字较少,故在训练集中加入了若干包含文字高分辨率图像,从而使得超分辨率模型对测试集低分辨率图像中的文字有了更好的重建效果。最终,本文通过严密消融实验,验证了上述改进方案的有效性和必要性。UBSR-DU模型在RealSR数据集的 Canon 4倍超分辨率测试图像中,在PSNR和SSIM指标上,分别比USR-DU模型提高了0.07dB和0.0044。

图像超分辨率技术旨在通过增加图像的分辨率来提高图像的清晰度和细节,使图像更为逼真,在摄影、监控、老旧图像重建等领域有着重要的应用。真实世界中的低分辨率图像往往由对应的高分辨率图像经复杂且未知的退化过程产生,这使得获取成对的训练集以训练超分辨率模型变得费力且高成本。近年来,研究者提出了许多基于无监督学习的超分辨率方法,即结合图像退化模型,使用非成对数据集训练一个降采样网络,并通过此网络大量生成伪成对的数据集。使用生成的数据集即可对超分辨率模型进行监督式的训练,从而降低了成对数据集的采集成本,并提高了模型的泛化性。

近期,一份对图像退化过程不确定性进行建模的工作USR-DU取得了良好的效果,并在多个数据集中超过了当前最先进的模型。尽管取得了优秀成绩,该模型在模糊性较高的真实低分辨率图像上表现较差。为了解决此问题,本文提出了改进模型\textbf{UBSR-DU}。本文首先比较了USR-DU的降采样网络生成的低分辨率图像与真实图像之间的差距,发现该差距出于对图像退化中的模糊过程的忽视。为了将模糊信息融入到降采样网络中,本文分析了多种模糊核估计方法,综合考虑其设计思想与估计效果,最后采用了基于空间变化核估计的MANet。之后,设计了对训练批次中低分辨率图像块进行模糊核估计,并将模糊核应用于降采样网络的参考图像的生成过程中的训练流程,从而成功地在降采样网络中融入了模糊信息。此外,考虑到本文中超分网络的测试集中具有较多的文字,而训练集高分辨率图像中文字较少,故在训练集中加入了若干包含文字高分辨率图像,从而使得超分辨率模型对测试集低分辨率图像中的文字有了更好的重建效果。最终,通过严密消融实验,我们验证了上述改进方案的有效性和必要性。UBSR-DU模型在RealSR数据集的 Canon 4倍超分辨率测试图像中,在PSNR和SSIM指标上,分别比USR-DU模型提高了\textbf{0.07dB}和\textbf{0.0012}。
