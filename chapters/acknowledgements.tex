光阴飞逝,转眼间,随着毕业论文的完成,四年的大学本科生活就要结束了。毕业设计的顺利完成离不开老师的悉心指导,同学、朋友、家长的陪伴和支持。

首先,我要感谢我的毕设指导老师毋芳芳老师。无论是选题、思路和还是实验,乃至论文的撰写和修改的过程中,毋老师都一直耐心指导,孜孜不倦。此外,也要感谢在毕设过程中给予我莫大帮助的董伟生老师,关于毕设中的创新点,董老师和我进行了亲切的交流,并给予了很多有建设性的建议,董老师对我的鼓励和督促为我完成毕设提供了巨大的动力。这些支持必定也会让我在将来的研究生的学习和研究中受益匪浅。除了学习方面,毋老师和董老师对我的生活方面也非常关心,关注我的心理健康和生活条件,并在我心情低落时给予我莫大的支持,让我倍感温暖。在此,祝董老师和毋老师科研工作顺利,身体健康。

其次,要感谢我的同学、师兄师姐和朋友。在研究过程中,我所使用的工作的作者宁倩师姐、唐静竹师姐和方振轩师兄就模型的参数、数据、训练过程等方面给予了我很多的指导,并耐心解答了我的许多问题,帮我找出了我犯的错误。在实验的过程中,周星宇同学和万世杰同学关于代码的架构、调试中遇到的问题和我进行了热切的交流,让我能够写出更加高效、稳定、可扩展性的代码。在我因实验失败而苦恼时,我的同学和朋友们也分担了我的焦虑情绪。在此,向支持过我的同学、师兄师姐和朋友致以诚挚的谢意,祝各位前程似锦。

此外,也要感谢我的家人。感谢他们二十多年来对我的养育之恩,以及对我学业的大力支持和付出。感谢他们始终为我着想,能让我没有后顾之忧地专注于学习和研究之中。感谢他们对我的鼓励和教导,让我能够从哪里跌倒从哪里爬起。祝愿我的家人身体健康,万事如意,我一定不会辜负你们的期望。

最后,要感谢西安电子科技大学高性能计算校级公共平台所提供的算力支持,让我能够使用先进的计算资源对提出的模型进行实验和验证。

此外,我诚挚的感谢各位评阅论文的专家老师,感谢你们从百忙之中抽出时间对本文进行评阅。