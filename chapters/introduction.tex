\section{研究背景及意义}
随着现代科技的飞速发展,数字图像成为了人们获取信息、传达思想的重要载体。而高质量的数字图像可以给人们带来更加真实、清晰的视觉体验,或在监控、救灾等领域起到重要的作用。而实际应用中,由于图像采集设备、传输通道等原因,图像常常受到噪声、模糊、压缩、下采样等各种退化影响,使得其分辨率较低,缺乏细节信息,不利于人类的使用。因此在计算机视觉领域中,图像超分辨率(Single Image Super-resolution, SISR)技术成为了一个备受关注的研究方向。

图像超分辨技术的目的是通过将低分辨率图像提高分辨率,从而得到更加清晰、细节更加丰富的图像。早期的超分辨率技术主要研究合成图像的超分辨率,通常假设高分辨率图像经由高斯模糊与双三次下采样过程退化为低分辨率图像。传统的方法包括插值、稀疏表示、局部嵌入等,而借由深度学习技术,研究者已提出许多优秀的合成图像的超分辨率模型。但将这些模型应用于真实图像时,会发现效果严重下降。由于真实图像的退化可能是由多种且未知的因素引起的,图像超分辨率处理难度较大。因此,如何提高真实图像的分辨率成为了图像超分辨技术中的重要研究方向。

在真实图像超分辨率中,基于图像退化模型估计的真实图像超分辨方法因其能够通过模拟图像退化过程来获取更加真实的成对高、低分辨率图像数据集用以训练超分辨率网络,而备受关注。这种方法能够在不同程度的噪声和模糊条件下提供更加清晰、细节更加丰富的超分辨图像。因此,本文将研究基于图像退化模型估计的真实图像超分辨方法,探索一种更加高效、准确的图像超分辨技术。

本文将对图像退化过程进行建模,通过合理的退化过程,生成更加真实且多样性的低分辨率图像,并在实验中进行评估,验证其与真实图像的相似性,进而使得能够通过生成的低分辨率图像训练一个超分辨率网络,使其具有将真实的低分辨率图像重建为高分辨率图像的能力。该研究成果可以为图像超分辨领域的相关研究提供借鉴和参考,同时也可以为图像采集、传输等领域的研究提供更加深刻的思考和理解,有助于推动图像处理技术的发展。另外,随着人工智能技术的不断发展,图像超分辨技术也逐渐应用于各个领域,如摄影、视频监控、医疗影像等。因此,本文的研究也具有重要的实际应用价值。

综上所述,本文将探究基于图像退化模型估计的真实图像超分辨方法,旨在提高真实图像的超分辨率,并为图像处理技术的发展和应用提供借鉴和参考。该研究成果对于推动图像超分辨技术在各个领域的应用和发展具有重要意义。
\section{历史及现状}

\section{论文的主要内容及工作安排}