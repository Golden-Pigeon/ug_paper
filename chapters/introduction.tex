\section{研究背景及意义}
随着现代科技的飞速发展,数字图像成为了人们获取信息、传达思想的重要载体。而高质量的数字图像可以给人们带来更加真实、清晰的视觉体验,或在监控、救灾等领域起到重要的作用。而实际应用中,由于图像采集设备、传输通道等原因,图像常常受到噪声、模糊、压缩、下采样等各种退化影响,使得其分辨率较低,缺乏细节信息,不利于人类的使用。因此在计算机视觉领域中,图像超分辨率(Single Image Super-resolution, SISR)技术成为了一个备受关注的研究方向。

图像超分辨技术的目的是通过将低分辨率图像提高分辨率,从而得到更加清晰、细节更加丰富的图像。早期的超分辨率技术主要研究合成图像的超分辨率,通常假设高分辨率图像经由高斯模糊与双三次下采样过程退化为低分辨率图像。传统的方法包括插值、稀疏表示等,而借由深度学习技术,研究者已提出许多优秀的合成图像的超分辨率模型。但将这些模型应用于真实图像时,会发现效果严重下降。由于真实图像的退化可能是由多种且未知的因素引起的,图像超分辨率处理难度较大。因此,如何提高真实图像的分辨率成为了图像超分辨技术中的重要研究方向。

在真实图像超分辨率中,基于图像退化模型估计的真实图像超分辨方法因其能够通过模拟图像退化过程来获取更加真实的成对高、低分辨率图像数据集用以训练超分辨率网络,而备受关注。这种方法能够在不同程度的噪声和模糊条件下提供更加清晰、细节更加丰富的超分辨图像。因此,本文将研究基于图像退化模型估计的真实图像超分辨方法,探索一种更加高效、准确的图像超分辨技术。

本文将对图像退化过程进行建模,通过合理的退化过程,生成更加真实且多样性的低分辨率图像,并在实验中进行评估,验证其与真实图像的相似性,进而使得能够通过生成的低分辨率图像训练一个超分辨率网络,使其具有将真实的低分辨率图像重建为高分辨率图像的能力。该研究成果可以为图像超分辨领域的相关研究提供借鉴和参考,同时也可以为图像采集、传输等领域的研究提供更加深刻的思考和理解,有助于推动图像处理技术的发展。另外,随着人工智能技术的不断发展,图像超分辨技术也逐渐应用于各个领域,如摄影、视频监控、医疗影像等。因此,本文的研究也具有重要的实际应用价值。

综上所述,本文将探究基于图像退化模型估计的真实图像超分辨方法,旨在提高真实图像的超分辨率,并为图像处理技术的发展和应用提供借鉴和参考。该研究成果对于推动图像超分辨技术在各个领域的应用和发展具有重要意义。
\section{历史及现状}
\subsection{传统的超分辨率技术}
图像超分辨率技术是一项具有悠久历史的技术,由于其广泛的应用场景,许多早期的研究人员已经从传统的图像处理技术或者机器学习技术出发提出了许多方法。早期的方法包括插值、稀疏表示、局部嵌入等,这些方法具有图像重建速度快的优势,但在图像质量上,往往会具有较多的锯齿、伪影,对图像细节的还原能力较差,不能与现今基于神经网络的模型相比。

\subsection{基于深度学习的超分辨率}
随着深度学习技术在计算机视觉领域崭露头角和高算力计算设备的问世,自2014年起,基于深度学习的模型逐渐成为图像超分辨率技术的主流。早期的模型通过将传统方法中的稀疏编码抽象为卷积操作,从而设计出了基于卷积神经网络的超分辨率模型。使用卷积神经网络后,不仅使得训练过程中模型收敛速度更快,降低训练时间成本,也使得图像在重建精度上显著提升。后续的工作中,研究者们从模型深度、残差连接、密集(dense)连接等角度对基于卷积神经网络的模型进行了改进优化,进一步提升了其超分辨率质量。

\subsection{真实图像的超分辨率}
早期的超分辨率模型往往假设高分辨率图像是通过高斯模糊和双三次下采样退化为低分辨率图像。当将这些模型应用于真实场景下时,会发现超分辨率效果显著下降。这是由于真实高分辨率图像是通过噪声、模糊、图像压缩等未知且复杂的方式退化为低分辨率图像,导致真实的低分辨率图像与合成的低分辨率图像在纹理细节、模糊程度等方面完全不一致。为了解决真实图像超分辨率的问题,一个显而易见解决方式是使用真实成对的数据集进行训练。然而,真实成对的数据集采集过程中往往需要耗费大量的人力物力财力,且一般体量不大。故许多研究人员考虑使用无监督学习技术来训练超分辨率网络。基于无监督学习的超分辨率模型可以分为两种类型:其一利用生成对抗模型(GAN),让生成器生成类似于真实高分辨率(清晰)图像的图像,而利用鉴别器判别生成图像和真实高分辨率图像(非成对)的真实性;另一种则从如何获取成对数据集入手,先训练一个降采样网络,用以生成拟真的低分辨率图像,以组成伪成对的数据集,再使用此成对数据集训练一个超分辨率网络。本文基于的工作USR-DU即为第二种类型,USR-DU模型在此基础上加入了不确定性学习,让模型的退化拥有了不确定性,从而能够生成更合理的低分辨率图像。然而,由于USR-DU的降采样部分没有充分利用真实低分辨率图像的模糊信息,故在较为模糊的图像域上不能生成较为合理的拟真低分辨率图像。

\section{论文的主要内容及工作安排}
本论文主要目的是解决USR-DU模型在模糊信息的利用上不足导致对较模糊的低分辨率图像重建能力较差的问题。为此提出了若干条改进方案,形成了改进模型UBSR-DU。

第一章阐述了基于退化模型估计的真实图像超分辨这一研究目标的意义,介绍了有关图像超分辨率这一领域的发展历史和相关工作。

第二章介绍了超分辨率任务评价指标、超分辨率模型相关工作、图像模糊核估计方法、超分辨率任务数据集等本文所使用的相关技术。

第三章通过实验解释了USR-DU具有什么样的缺陷,并指出这种缺陷来自于模型的DSN部分对真实低分辨率图像模糊信息的忽略。进而提出了引入真实图像模糊信息的可选方案,通过实验得出了最佳改进方案。随后,经过详细的消融实验验证了此方案的有效性、可靠性和必要性。最后,比较了提出的模型与现有最先进的模型之间的效果、性能差异。

第四章对全文工作进行了总结,给出了所提出模型的局限性,并给出了可行的进一步研究方向。