\section{局限性分析}
虽然本文提出的模型UBSR-DU在多个数据集上均取得了明显优于现阶段最优模型的表现,但其仍然存在不少缺陷,现列举如下
\begin{enumerate}
    \item 模型生成的超分图像在感知质量上仍有不足,经过对LPIPS这一指标(此指标通过卷积神经网络学习了人类对图像的感知差异)的评估,发现该模型劣于USR-DU模型。
    \item 由于DSN部分包含了较复杂的过程,故模型的训练较为缓慢,这给调整合适的参数带来了很大的困难。
    \item DSN目前仍然只能生成训练集域内的低分辨率图像,而无法通过额外指定参数生成其他域的低分辨率图像,这降低了模型的泛化性和可用性。
\end{enumerate}
\section{下一步研究内容}
\begin{enumerate}
    \item 由于本模型基于对图像退化过程进行建模和估计,这要求训练集的高低分辨率图像块具有相似的内容丰富程度。由于图像块实际上是在数据集中随机裁剪产生的,故可能出现高分辨率块内容丰富而低分辨率块内容较为平淡的情况,反之亦然,这导致估计出的退化模型的稳定性较差,以至于某些batch可能使模型产生负向优化。故未来可以基于训练batch中高低分辨率图像块的内容丰富程度设计一个可学习的参数,并在损失函数中引入此参数,以权衡不同图像对对优化目标的贡献。
    \item 可以设法通过在DSN推理过程中给予一张或若干张参考图像,使得其能够生成符合参考图像域的低分辨率图像,这样就可以避免对每一个数据集均训练一个特别的DSN网络。
\end{enumerate}
\section{总结}
本文针对USR-DU的降采样网络部分对图像模糊生成不足的问题,基于图像退化模型估计对该模型进行改进,进而提出了基于不确定性学习的无监督的模糊真实世界图像超分辨率UBSR-DU这一改进模型。本文主要工作如下:
\begin{enumerate}
    \item 对USR-DU模型进行了分析,发现其对模糊性较高低分辨率图像超分辨率效果不佳,并指出该缺陷来自于降采样网络没有合理利用训练集中低分辨率图像的模糊信息。
    \item 为了引入模糊信息,比较了多种模糊核估计方法,最后选择了MANet作为本文采用的模糊核估计方法。
    \item 设计了模糊信息融入流程,并使用PyTorch进行了实现。同时,为加强模型对文字内容的重建能力,另外使用了RealSR-Nikon作为超分辨率网络训练集的一部分。通过与当前最先进模型的对比和充分的消融实验验证了所作改进的合理性。
\end{enumerate}